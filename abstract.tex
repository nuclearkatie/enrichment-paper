
\begin{abstract}
\begin{itshape}
This work compares and evaluates different computational approaches for modeling
off-normal operation of a gas centrifuge enrichment cascade.

The goal of this work focuses on developing the necessary understanding of
potential misuse of enrichment cascades, contributing to more effective
international safeguards designs and approaches.  While it is straightforward to
design a symmetric enrichment cascade under ideal conditions as a function of the
theoretical feed, product, and tails assays, it is very difficult to find
reliable information about the behavior of a given cascade when the feed assay
does not match the design value. Several methods have been developed to assess
the behavior of an enrichment cascade in such circumstances. In addition to the
cut, ($\theta$) these methods evaluate the feed-to-product, feed-to-tails, and the
product-to-tails enrichment ratio, $\alpha$, $\beta$ and $\gamma$, respectively,
as a function of the cascade feed assay. As those four parameters depend on each
other, determining two of them fully defines the other.  The first approach
consists of fixing $\theta$ and $\alpha$, recomputing the corresponding assays at
each stages of the cascade. The second one maintains the ideal condition of the
cascade ($\alpha$ and $\beta$ fixed across the whole cascade), modifying
$\theta$ values at each stage accordingly. Both approaches have been implemented
into the \Cyclus fuel cycle simulator\cite{cyclus, mbmore.2018}. The third fixes
$\theta$ and $\gamma$, using both $\alpha$ and $\beta$ at each stage as free
parameters. The third method has been investigated in \cite{walker.2017}.

Following a description of each method and an evaluation of differences between
each approach, this work compares the results produced by these methods within
scenarios involving misuse of symmetric enrichment cascades simulated using the dynamic
nuclear fuel cycle simulator, \Cyclus.
\end{itshape}

\end{abstract}
